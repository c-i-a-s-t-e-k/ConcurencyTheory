\documentclass{article}
\usepackage{graphicx}

\title{Mnożenie macierzy}
\author{Dominik Jeżów}
\author{Szymon Twardosz}
\date{\today}

\begin{document}

\maketitle

\section{Użyte środowisko i język}
Do wykonania ćwiczenia wykorzystaliśmy jęztk python 3.11 wraz z następujęcymi bibliotekami
numpy, matplotlib oraz time

\section{Temat zadania}
Zadanie polegało na zaimplementowaniu 3 różnych metod mnożenia macierzy kwadratowych o wymiarach będących potęgą liczby 2. Są to kolejno:
\begin{description}
    \item[Metoda Bin\grave{e}t'a] Najprostsza metoda rekurencyjna polegająca na obliczeniu macierzy wynikowej na podstawie 4 oddzielnych fragmentów
    \item[Metoda Strassena] Podobna do powyrzszej metody z tą różnicą że dzielimy macierz na więcej części aby zmniejszyć ilość wykonanych operacji arytmetycznych.
    \item[Metoda Alpha Tensor] Tak jak wyżej tylko, że ilość na którą została podzielona macierz została znaleziona przez model sztucznej inteligencji.
  \end{description}

\section{Wyniki}
\begin{description}
    \item[Metoda Bin\grave{e}t'a] Metoda ta otrzymała wynkiki w
    najszybszym czasie - dla macierzy kwadratowytch o wymiarze 1024 otrzymał wynik w czasie prawie 2 razy szybszym od metody Strassena.
    \item[Metoda Strassena] Opis drugiego elementu.
    \item[Metoda Alpha Tensor] Opis trzeciego elementu.
  \end{description}

\subsection{Porównanie czasów wykonania}
\begin{figure}
    \centering
    \includegraphics[width=0.6\textwidth]{output2.png}
    \caption{Opis obrazu}
    \label{fig:obraz}
\end{figure}

\subsection{Porównanie wykonanych operacji arytmetycznych}
\begin{figure}
    \centering
    \includegraphics[width=0.6\textwidth]{output.png}
    \caption{Opis obrazu}
    \label{fig:obraz}
\end{figure}

\section{Wnioski}
Mimo że metoda Alpha Tensor oraz Strassena wymagoją mniejszej liczby operacji algorytmicznej to poprzez jej niefektywnej implementacji nie osiągają one leprzych wyników czasowych co metoda Bin\grave{e}t'a 
\end{document}
